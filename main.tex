\documentclass[12pt]{report}
\usepackage[letterpaper]{geometry}
\usepackage[spanish, es-lcroman, es-noshorthands]{babel}
\usepackage[utf8]{inputenc}
\usepackage{lmodern}
\usepackage{amsmath, amsthm, amssymb}
\usepackage{graphicx}
\usepackage{caption}

\geometry{twoside, 
  paperheight = 24.0cm,
  paperwidth = 17.0cm,
  columnsep = 1.0cm,
  textheight = 19.0cm,
  textwidth = 12.8cm,
  centering,
  marginparwidth = 1cm,
  top = 3.0cm}

\captionsetup[table]{
    format=plain,
    labelformat=simple,
    labelsep=colon,
    justification=centering,
    singlelinecheck=false,
    textfont={bf}
}

\begin{document}

% Carátula
\begin{titlepage}
    \centering
    {\huge\bfseries Informe Esstadístihcomm de Encuesta Lenguajes Preferidos de Programación \par}
    \vspace{1cm}
    {\Large Nombre del Curso \par}
    \vspace{2cm}
    {\Large Nombre del Alumno \par}
    \vfill
    {\large Fecha \par}
    \vfill
    {\large Universidad \par}
\end{titlepage}

% Tabla de contenidos
\tableofcontents
\newpage

% Capítulos
\chapter{Introduccións}
\input{Capitulos/introduccion.tex}

\chapter{Metodología}
\input{Capitulos/metodologia.tex}

\chapter{Resultados}
\input{Capitulos/resultados.tex}

\chapter{Conclusión}
\input{Capitulos/conclusion.tex}

% Ejemplo de tabla
\begin{table}[h!]
    \centering
    \caption{\\Tabla de Distribución de Frecuencia \\ Sexo \\ Carrera - Informática \\ FCPN - UMSA}
    \label{tab:distribucion_frecuencia}
    \begin{tabular}{|c|c|c|}
        \hline
        Sexo      & ni & fi \\ \hline
        Masculino & 12 & 50 \\ \hline
        Femenino  & 13 & 30 \\ \hline
    \end{tabular}
    \vspace{0.5cm}
    \caption*{Fuente: Elaboración propia (invierno-2024)}
\end{table}

\end{document}
